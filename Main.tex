\documentclass[12pt]{article}

%\usepackage[utf8]{inputenc}
\usepackage{indentfirst}
\usepackage[hidelinks]{hyperref}
\usepackage{float}
\usepackage{array}
\usepackage{listings}
\usepackage{csquotes}
\usepackage{hyperref}
\usepackage{enumitem, amsmath, amssymb, amsfonts, latexsym, mathrsfs}
\usepackage{graphicx}
\usepackage{subfig}
%\usepackage[greek,english]{babel}
%\usepackage{alphabeta}
\usepackage{multicol}

\newlist{todolist}{itemize}{2}
\setlist[todolist]{label=$\square$}
\usepackage{pifont}
\newcommand{\cmark}{\ding{51}}%
\newcommand{\xmark}{\ding{55}}%
\newcommand{\done}{\rlap{$\square$}{\raisebox{2pt}{\large\hspace{1pt}\cmark}}%
\hspace{-2.5pt}}
\newcommand{\wontfix}{\rlap{$\square$}{\large\hspace{1pt}\xmark}}

\date{}
% Comand para keywords
\providecommand{\keywords}[1]
{
  \small
  \textbf{\textit{Keywords---}} #1
}

% Setup de hiperenlaces
\hypersetup{
    colorlinks=true,
    linkcolor=cyan,
    filecolor=magenta,
    urlcolor=cyan,
    pdftitle={GodOfJustice},
    pdfpagemode=FullScreen,
    }

% Tipografía
\usepackage{helvet}
\renewcommand{\familydefault}{\sfdefault}
\usepackage[sfdefault]{inter}
\usepackage{comment}

% Listas
%\newlist{todolist}{itemize}{2}
%\setlist[todolist]{label=$\square$}

% Imagenes
\graphicspath{ {./images/} }

% Interlineado
\usepackage{setspace}
\spacing{1.5}

% Márgenes
\usepackage[a4paper]{geometry}
\geometry{top=2.5cm, bottom=2.5cm, left=2cm, right=2cm}

% Número de página
\usepackage{fancyhdr}
\pagestyle{fancy}
\rhead[]{}
\lhead[]{}
\renewcommand{\headrulewidth}{0pt}
\rfoot[]{\thepage}
\cfoot[]{}

%_____________________________________________________________________________
%_____________________________________________________________________________
%_____________________________________________________________________________
%_____________________________________________________________________________
\begin{document}

% PORTADA
    \begin{titlepage}

        \centering
        \hrule
        %\vspace{1cm}
        %{\bfseries\Large UNIVERSIDAT JAUME I \par}
        \vspace{1cm}
        {\bfseries\huge Analisis Artistico de the legend of Zelda \par}
        \vspace{3cm}
        {\includegraphics[width=0.7\textwidth]{UJI_logo.jpg} \par}
        \vspace{4cm}
        {\LARGE \textbf{Deadlyrup} \par}

        {\large
        Nerea Villarrolla Marco \\
        Saul Pacheco Trilles \\
        Alonso Madrigal Hernández \\
        Carlos Castell Campos \\
        Jesus Jimenez Montero \\
        Raul Montero Piñeiro \\
        Selena Monforte Arques \\
        Joan \\
        \par}
        \vspace{10cm}
        \hrule

    \end{titlepage}

 % abstract
\newpage
\begin{abstract}
    En el siguiente documento hablaremos sobre las bases artisticas del videojuego "The Legend Of Zelda: Breath of the wild". Tanto del director de concept art como un atisbo de información sobre el juego. Finalemnete acabaremos con el analisis artistico profundo de varios fan arts.

\end{abstract}

\keywords{Videojuego, arte, concept art, analisis}

% ÍNDICE
%\renewcommand{\tableofcontents}{Indice general}
\newpage
\tableofcontents
\setcounter{tocdepth}{4}

\newpage
%-----------------------------------------------------------------
%-----------------------------------------------------------------
% Tabla de figuras
\newpage
\renewcommand{\listfigurename}{Lita de figuras}
\thispagestyle{empty}
\listoffigures

%-----------------------------------------------------------------
%-----------------------------------------------------------------

\newpage
\section{Introducción}

\section{El juego}

\subsection{Satoru Takizawa}

\section{Relaciones Conceptuales}

\subsection{Arquitectura}

\subsection{Arte}

\subsection{Fotografía}

\subsection{La música}


%-----------------------------------------------------------------
%-----------------------------------------------------------------
%

\section{Analisis conceptual de imágenes}
    \subsection{1. Nerea}
        \subsubsection{Perspectiva}
    
        \subsubsection{Composición}
    
        \subsubsection{Clarooscuro}
    
        \subsubsection{Color}
    \subsection{2. Jesús}
        \subsubsection{Perspectiva}
    
        \subsubsection{Composición}
    
        \subsubsection{Clarooscuro}
    
        \subsubsection{Color}
    \subsection{3. Saul}
        \subsubsection{Perspectiva}
    
        \subsubsection{Composición}
    
        \subsubsection{Clarooscuro}
    
        \subsubsection{Color}
    \subsection{4. Raúl}
        \subsubsection{Perspectiva}
    
        \subsubsection{Composición}
    
        \subsubsection{Clarooscuro}
    
        \subsubsection{Color}
    \section{5. Miquel}
        \subsubsection{Perspectiva}
    
        \subsubsection{Composición}
    
        \subsubsection{Clarooscuro}
    
        \subsubsection{Color}
    \subsection{6. Nerea (segunda imagen)}
        \subsubsection{Perspectiva}
    
        \subsubsection{Composición}
    
        \subsubsection{Clarooscuro}
    
        \subsubsection{Color}
    \subsection{7. Miquel (segunda imagen)}
        \subsubsection{Perspectiva}
    
        \subsubsection{Composición}
    
        \subsubsection{Clarooscuro}
    
        \subsubsection{Color}
    \subsection{8. Raúl (Segunda imagen)}
        \subsubsection{Perspectiva}
    
        \subsubsection{Composición}
    
        \subsubsection{Clarooscuro}
    
        \subsubsection{Color}
    \subsection{9. Alonso}
        \subsubsection{Perspectiva}
    
        \subsubsection{Composición}
    
        \subsubsection{Clarooscuro}
    
        \subsubsection{Color}
    \subsection{10. Selena}
        \subsubsection{Perspectiva}
    
        \subsubsection{Composición}
    
        \subsubsection{Clarooscuro}
    
        \subsubsection{Color}
    \subsection{11. Alonso (Segunda Imagen)}
        \subsubsection{Perspectiva}
    
        \subsubsection{Composición}
    
        \subsubsection{Clarooscuro}
    
        \subsubsection{Color}
    \subsection{12. Carlos}
        \subsubsection{Perspectiva}
    
        \subsubsection{Composición}
    
        \subsubsection{Clarooscuro}
    
        \subsubsection{Color}
    \subsection{13. Joan}
        \subsubsection{Perspectiva}
    
        \subsubsection{Composición}
    
        \subsubsection{Clarooscuro}
    
        \subsubsection{Color}
    \subsection{14. Selena (Segunda imagen)}
        \subsubsection{Perspectiva}
    
        \subsubsection{Composición}
    
        \subsubsection{Clarooscuro}
    
        \subsubsection{Color}
    \subsection{15. Carlos (Segunda Imagen)}
        \subsubsection{Perspectiva}
    
        \subsubsection{Composición}
    
        \subsubsection{Clarooscuro}
    
        \subsubsection{Color}
    \subsection{16. Jesus}
        \subsubsection{Perspectiva}
    
        \subsubsection{Composición}
    
        \subsubsection{Clarooscuro}
    
        \subsubsection{Color}
    \subsection{17. Joan (Segunda Imagen)}
        \subsubsection{Perspectiva}
    
        \subsubsection{Composición}
    
        \subsubsection{Clarooscuro}
    
        \subsubsection{Color}
    \subsection{18. Jesus (Segunda Imagen)}
        
\end{document}