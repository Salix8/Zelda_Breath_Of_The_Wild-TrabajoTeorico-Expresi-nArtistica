\documentclass[12pt]{article}

%\usepackage[utf8]{inputenc}
\usepackage{indentfirst}
\usepackage[hidelinks]{hyperref}
\usepackage{float}
\usepackage{array}
\usepackage{listings}
\usepackage{csquotes}
\usepackage{hyperref}
\usepackage{enumitem, amsmath, amssymb, amsfonts, latexsym, mathrsfs}
\usepackage{graphicx}
\usepackage{subfig}
%\usepackage[greek,english]{babel}
%\usepackage{alphabeta}
\usepackage{multicol}

\newlist{todolist}{itemize}{2}
\setlist[todolist]{label=$\square$}
\usepackage{pifont}
\newcommand{\cmark}{\ding{51}}%
\newcommand{\xmark}{\ding{55}}%
\newcommand{\done}{\rlap{$\square$}{\raisebox{2pt}{\large\hspace{1pt}\cmark}}%
\hspace{-2.5pt}}
\newcommand{\wontfix}{\rlap{$\square$}{\large\hspace{1pt}\xmark}}

\date{}
% Comand para keywords
\providecommand{\keywords}[1]
{
  \small
  \textbf{\textit{Keywords---}} #1
}

% Setup de hiperenlaces
\hypersetup{
    colorlinks=true,
    linkcolor=cyan,
    filecolor=magenta,
    urlcolor=cyan,
    pdftitle={GodOfJustice},
    pdfpagemode=FullScreen,
    }

% Tipografía
\usepackage{helvet}
\renewcommand{\familydefault}{\sfdefault}
\usepackage[sfdefault]{inter}
\usepackage{comment}

% Listas
%\newlist{todolist}{itemize}{2}
%\setlist[todolist]{label=$\square$}

% Imagenes
\graphicspath{ {./images/} }

% Interlineado
\usepackage{setspace}
\spacing{1.5}

% Márgenes
\usepackage[a4paper]{geometry}
\geometry{top=2.5cm, bottom=2.5cm, left=2cm, right=2cm}

% Número de página
\usepackage{fancyhdr}
\pagestyle{fancy}
\rhead[]{}
\lhead[]{}
\renewcommand{\headrulewidth}{0pt}
\rfoot[]{\thepage}
\cfoot[]{}

%_____________________________________________________________________________
%_____________________________________________________________________________
%_____________________________________________________________________________
%_____________________________________________________________________________
\begin{document}

% PORTADA
    \begin{titlepage}

        \centering
        \hrule
        %\vspace{1cm}
        %{\bfseries\Large UNIVERSIDAT JAUME I \par}
        \vspace{1cm}
        {\bfseries\huge Analisis Artistico de the legend of Zelda \par}
        \vspace{3cm}
        {\includegraphics[width=0.7\textwidth]{images/UJI_logo.jpg} \par}
        \vspace{4cm}
        %{\LARGE \textbf{Deadlyrup} \par}

        {\large
        Nerea Villarrolla Marco \\
        Saul Pacheco Trilles \\
        Alonso Madrigal Hernández \\
        Carlos Castell Campos \\
        Jesus Jimenez Montero \\
        Raul Montero Piñeiro \\
        Selena Monforte Arques \\
        Joan \\
        \par}
        \vspace{10cm}
        \hrule

    \end{titlepage}

 % abstract
\newpage
\begin{abstract}
    En el siguiente documento hablaremos sobre las bases artisticas del videojuego "The Legend Of Zelda: Breath of the wild". Tanto del director de concept art como un atisbo de información sobre el juego. Finalemnete acabaremos con el analisis artistico profundo de varios fan arts.

\end{abstract}

\keywords{Videojuego, arte, concept art, analisis}

% ÍNDICE
%\renewcommand{\tableofcontents}{Indice general}
\newpage
\tableofcontents
\setcounter{tocdepth}{4}

\newpage
%-----------------------------------------------------------------
%-----------------------------------------------------------------
% Tabla de figuras
\newpage
\renewcommand{\listfigurename}{Lita de figuras}
\thispagestyle{empty}
\listoffigures

%-----------------------------------------------------------------
%-----------------------------------------------------------------

\newpage
\section{Introducción}
    \hrule
\vspace{1cm}
    En el siguiente documento queda expuesto nuestro trabajo a lo largo de la asignatura: VJ1204. El cual ha sido un analisis profundo de las diferentes caracteristicas y aspectos que conlleva realizar una composición. Tendremos en cuenta todos estos puntos:
    \begin{itemize}
        \item Perpectiva; línea de horzionte, tupo de vista y puntos de fuga.
        \item Composición; regla de los 2/3, puntos de interes, recorridos visuales, ley de la balanza y simetria.
        \item Claroscuro; clave de la imagen, zonas de iluminación, profundidad, fuente de luz.
        \item Color; gama de colores, tonalidad general, tipos de contraste y colores empleados en primer y segundo plano.
    \end{itemize}

    En una primera parte daremos una pequeña introducción sobre el juego además de dar mención del director de arte principal, el cual fue el que creo los primeros concepts arts de tanto el juego como los personajes. Despues pasaremos ha hablar sobre algunas relaciones conceptuales de diferentes disciplinas, pues en el juego se pueden ver diferentes tipos de arquitecturas que hacen referencia a algunas de la vida real, metodos de arte en el juego que brinda al mundo de hyrule una personalidad única...
\section{El juego}
    \hrule
\vspace{1cm}
The Legend Of Zelda: Breath of the wild es la décima octava entrega de la saga The Legend of Zelda. Lanzado en el 2017 por Nintendo en la consola Nintendo switch y la WII U, esta entrega nos brinda un mundo abierto lleno de cosas por hacer. Podemos ir directamente por el jefe final, o podemos perdernos por el colosal e interesante mundo. Yendo a los sitios más reconditos e intrincados de llegar.
    El mundo esta embadurnado de secretos, personajes interesantes, misterios por descubrir y batallas que librar. Este titlo ha sido aclamado por la critica desde su salida, siendo para muchos uno de los mejores titulos que hayan podido jugar. Pero sus afirmaciones no son fundamentos sin una base solida y es que Breath of the wild, tiene notas que no bajan del nueve por más de 14 revistas. Ganó 5 premios entre ellos el GOTY y mejor dirección de juego. Y es que este juego ha sido un punto y aparte en los videojuegos, un ejemplo a seguir de como ha de realizarse un mundo abierto.

\subsection{Satoru Takizawa}
Comenzó creando el logo de Yoshi's Island 2, pero acabo creando las bases más importantes de la saga Zelda, y e que Satoru ha creado el diseño del proprio Ganondorf, el villano principal de toda la saga. También ha dado vida a varios enemigos menores de la saga Mario bros que se han vuelto marca insignia de los juegos del fontanero. Para el titulo The Legend of Zelda: Twilight Princess, Satoru fue nombrado director de arte del juego, siendo este su primer trabajo en este rol. Despues debido a su excelente trabajo se siguió nombrando a el para desempeñar este rol, siendo uno de esos trabajos The Legend of Zelda: Breath of the Wild.
\section{Relaciones Conceptuales}
    \hrule
\vspace{1cm}

\subsection{Arquitectura}

\subsection{Arte}

\subsection{Fotografía}

\subsection{La música}
Se tiene en cuenta que la mayoría de las travesías en la aventura, nos invade el silencio o los sonidos ambientales de los pájaros o más animales. Pero la música de ZBoW juega un gran papel en lo que es el game feel. Ya que en este juego a pesar de momentos puntuales no hace uso de grandes orquestas filarmónicas, si no de serenas melodías que nos ayudan a situarnos tanto geográficamente como temporalmente. Suaves flautas y oboes en la ciudad gerudo con el toque desértico que nos evocan a montañas de arena. Música con fuerte repercusión en la ciudad Goron, pero ninguna de estas opacando la acción del protagonista, si no más bien acompañándolo serenamente.

%-----------------------------------------------------------------
%-----------------------------------------------------------------


\section{Analisis conceptual de imágenes}
    \hrule
\vspace{1cm}
    \subsection{1. Nerea}
        \subsubsection{Perspectiva}

        \subsubsection{Composición}

        \subsubsection{Clarooscuro}

        \subsubsection{Color}
        \newpage

%-----------------------------------------------------------------
%-----------------------------------------------------------------

    \subsection{2. Jesús}
        \subsubsection{Perspectiva}

        \subsubsection{Composición}

        \subsubsection{Clarooscuro}

        \subsubsection{Color}
        \newpage

%-----------------------------------------------------------------
%-----------------------------------------------------------------

    \subsection{3. Saul}
        \subsubsection{Perspectiva}

        \subsubsection{Composición}

        \subsubsection{Clarooscuro}

        \subsubsection{Color}
        \newpage

%-----------------------------------------------------------------
%-----------------------------------------------------------------

    \subsection{4. Raúl}
        \subsubsection{Perspectiva}

        \subsubsection{Composición}

        \subsubsection{Clarooscuro}

        \subsubsection{Color}
        \newpage

%-----------------------------------------------------------------
%-----------------------------------------------------------------

    \section{5. Miquel}
        \subsubsection{Perspectiva}

        \subsubsection{Composición}

        \subsubsection{Clarooscuro}

        \subsubsection{Color}
        \newpage

%-----------------------------------------------------------------
%-----------------------------------------------------------------

    \subsection{6. Nerea (segunda imagen)}
        \subsubsection{Perspectiva}

        \subsubsection{Composición}

        \subsubsection{Clarooscuro}

        \subsubsection{Color}
        \newpage

%-----------------------------------------------------------------
%-----------------------------------------------------------------

    \subsection{7. Miquel (segunda imagen)}
        \subsubsection{Perspectiva}

        \subsubsection{Composición}

        \subsubsection{Clarooscuro}

        \subsubsection{Color}
        \newpage
%-----------------------------------------------------------------
%-----------------------------------------------------------------

    \subsection{8. Raúl (Segunda imagen)}
        \subsubsection{Perspectiva}

        \subsubsection{Composición}

        \subsubsection{Clarooscuro}

        \subsubsection{Color}
        \newpage

%-----------------------------------------------------------------
%-----------------------------------------------------------------

    \subsection{9. Alonso}
    \begin{figure}[H]
      \centering
      \includegraphics[scale=0.1]{images/Concepts/9_concept_art}
      \caption{\small Portada de la foto}
    \end{figure}
    Se trata de una imagen oficial del juego, la portada del mismo.

        \subsubsection{Perspectiva}

    \begin{figure}[H]
      \centering
      \includegraphics[scale=0.1]{images/Alonso/Sección 9/Linea orizonte}
      \caption{\small Linea del horizonte}
    \end{figure}

    Comenzaremos hablando de la línea del horizonte la cual está situada justo en medio de la imagen. Se puede discernir perfectamente debido a que se encuentra directamente el propio horizonte en la lejanía delimitado por una línea perfectamente recta. En función de estos datos podemos concluir  que se trata de una vista serena.

    \begin{figure}[H]
      \centering
      \includegraphics[scale=0.1]{images/Alonso/Sección 9/caja.jpg}
      \caption{\small puntos de fuga}
    \end{figure}

    Cómo está estructurada la imagen no podremos sacar información de la misma. Los elementos son demasiado orgánicos como para establecer la existencia de puntos de fuga que nos aclaren la perspectiva. Sin embargo si cerramos la piedra en la que se encuentra link en una caja. Sacamos más información pues así dibujado podríamos decir que hay más de un punto de fuga. Aun así la representación de la caja puede llegar a ser demasiado subjetiva. Es por ello que no se puede establecer con exactitud el número de puntos de fuga ni su ubicación.

        \subsubsection{Composición}
    \begin{figure}[H]
      \centering
      \includegraphics[scale=0.1]{images/Alonso/Sección 9/2-3.jpg}
      \caption{\small 2/3}
    \end{figure}
    Con la regla de los 2/3 podremos ver los puntos más notorios de la imagen, pues son aquellos que caen exactamente en las esquinas de los cuadrados que representa esta regla. Señalando y dando protagonismo a 2 puntos de interés. Que prácticamente serían los dos únicos más importantes en toda la imagen.

    \begin{figure}[H]
      \centering
      \includegraphics[scale=0.1]{images/Alonso/Sección 9/Puntos de interes.jpg}
      \caption{\small puntos de interes}
    \end{figure}

    Los puntos de interés, gracias a la regla de los 2/3 podremos señalar tanto el sol que está iluminando a todo Hyrule, es decir, a todos los elementos de la imagen, siendo el propio sol el elemento más iluminado. Y al propio link que se encuentra en la segunda mitad de toda la composición y en un primer plano dándole protagonismo a sí mismo.
    Y son pues estos elementos los primeros a los que acude la mirada en toda la representación. También podríamos marcar como puntos de interés secundarios el castillo de hyrule, el volcán y los pájaros en la parte superior de la imagen. El castillo por tener un destello que marca claramente un punto de interés. Y el volcán señalando con su estructura natural el sol que ilumina toda la escena, acompañado de los pájaros que realizan la misma función.

    \begin{figure}[H]
      \centering
      \includegraphics[scale=0.1]{images/Alonso/Sección 9/recorrido visual.jpg}
      \caption{\small recorrido visual}
    \end{figure}

    Es por ello que el recorrido visual es más claro, Pues los pájaros, el volcán, la propia estructura del castillo de hyrule, la mirada de link y su posición, prácticamente todos los elementos de la composición. Nos dan a entender que el elemento principal es el destello del sol. Mires el elemento que mires te da a entender que hay un elemento más importante y es por ello que te lo está señalando con su propia estructura.

    \begin{figure}[H]
      \centering
      \includegraphics[scale=0.1]{images/Alonso/Sección 9/ley de la balanza.jpg}
      \caption{\small Balanza y simetría}
    \end{figure}

    La ley de la balanza se cumple de forma espléndida, porque ninguno de los elementos principales de la imagen opacan al otro en protagonismo, si no que se podrían considerar igual de notorios e importantes. Pues la luminosidad del potente sol está compensada con la imponente puesta en escena de Link, posando con su espada mirando al horizonte, apoyado con la altura de la roca. Manteniendo los dos elementos la misma cantidad de peso. Con ello se puede aprovechar para decir que no existe ninguna simetría en la imagen, por lo que es asimétrica.

        \subsubsection{Clarooscuro}

    \begin{figure}[H]
      \centering
      \includegraphics[scale=0.1]{images/Alonso/Sección 9/blanco y negro.jpg}
      \caption{\small Blanco y negro}
    \end{figure}
    Con la imagen en blanco y negro podemos hacer un rapido analisis con sus profundidades, se ven claramente 2 profundidades demarcadas de forma muy notoria con la luminosidad, pues las partes en primer plano tienen una clave más baja que la que estan en segundo plano, que adquieren más protagonismo por estar tan iluminadas.

    \begin{figure}[H]
      \centering
      \includegraphics[scale=0.1]{images/Alonso/Sección 9/pixel.jpg}
      \caption{\small Pixel}
    \end{figure}

    Sin embargo cuando pixelamos la imagen nos podemos dar más cuenta de que debido al contraste de luminosidad entre el primer y segundo plano hace una mezcla de valores de color que hace detonar la imagen con una clave media. Pues ninguno de los dos valores de luminosidad resalta sobre el otro.

    \begin{figure}[H]
      \centering
      \includegraphics[scale=0.1]{images/Alonso/Sección 9/luminosidad.jpg}
      \caption{\small Fuente de luz}
    \end{figure}

    Es por el elemento principal de la imagen la causa de la fuente de luminosidad, pues debido a su condición de ser un sol está iluminando a toda la imagen, dejando solo oscura las sombras que provoca esta misma iluminación. Marcadas perfectamente en la imagen.

        \subsubsection{Color}
        \begin{figure}[H]
      \centering
      \includegraphics[scale=0.1]{images/Alonso/Sección 9/Paleta color.jpg}
      \caption{\small paleta de color}
    \end{figure}

        La gama cromática empleada ha resultado finalmente tener una estructura bastante lógica, pues se trata de una triada de colores, debido al verde de la hierba, el azul del cielo y del reino de hyrule y el sol anaranjado. Dan como resultado una triada de colores bastante definidos.
        \begin{figure}[H]
      \centering
      \includegraphics[scale=0.1]{images/Alonso/Sección 9/colores.jpg}
      \caption{\small Tonalidad general}
    \end{figure}
    La tonalidad global de la imagen es difícil de determinar, debido a que gracias a su iluminación tan potente, hace un macro contraste de tono entre una tonalidad fría y una tonalidad algo más cálida. Lo que sí podemos describir es la saturación general de los colores presentes en la imagen. Pues el tono es muy intenso dejando una saturación  bastante alta en prácticamente toda la imagen.

    \begin{figure}[H]
      \centering
      \includegraphics[scale=0.1]{images/Alonso/Sección 9/Contrastes.jpg}
      \caption{\small tipos de contrastes}
    \end{figure}
    Los tipos de contraste en esta imagen son bastante claros en cada uno de sus apartados. Empezando con el Tono, hay un intenso contraste de tonos a lo largo de la representación. Principalmente se trata del contraste entre la luz anaranjada del sol y el azul intenso del cielo. Gracias a la sombra de la piedra provocada por el sol, podemos ver el contraste de luminosidad entre la sombra y la parte iluminada por el foco de luz. Y finalmente hay un ligero contraste de Saturación, pues a medida que nos acercamos al horizonte podemos ver un color mucho más apagado que el del centro del foco.

    \begin{figure}[H]
      \centering
      \includegraphics[scale=0.1]{images/Alonso/Sección 9/analisis de meirda.jpg}
      \caption{\small Analisis de capas}
    \end{figure}

    Por todo lo comentado podemos concluir lo siguiente: Entre el primer y segundo plano hay un contraste de luminosidad y contrastes bastante evidente. El primer plano tiene una luminosidad apagada con una saturación alta, pero poco notable debido a su poca luminosidad. Y en suegundo plano si que podemos ver colores vivos e iluminados, pero con la misma cantidad de saturación.


        \newpage

%-----------------------------------------------------------------
%-----------------------------------------------------------------

    \subsection{10. Selena}
        \subsubsection{Perspectiva}

        \subsubsection{Composición}

        \subsubsection{Clarooscuro}

        \subsubsection{Color}
        \newpage

%-----------------------------------------------------------------
%-----------------------------------------------------------------

    \subsection{11. Alonso (Segunda Imagen)}
        \subsubsection{Perspectiva}

        \subsubsection{Composición}

        \subsubsection{Clarooscuro}

        \subsubsection{Color}
        \newpage

%-----------------------------------------------------------------
%-----------------------------------------------------------------

    \subsection{12. Carlos}
        \subsubsection{Perspectiva}

        \subsubsection{Composición}

        \subsubsection{Clarooscuro}

        \subsubsection{Color}
        \newpage

%-----------------------------------------------------------------
%-----------------------------------------------------------------

    \subsection{13. Joan}
        \subsubsection{Perspectiva}

        \subsubsection{Composición}

        \subsubsection{Clarooscuro}

        \subsubsection{Color}
        \newpage

%-----------------------------------------------------------------
%-----------------------------------------------------------------

    \subsection{14. Selena (Segunda imagen)}
        \subsubsection{Perspectiva}

        \subsubsection{Composición}

        \subsubsection{Clarooscuro}

        \subsubsection{Color}
        \newpage

%-----------------------------------------------------------------
%-----------------------------------------------------------------

    \subsection{15. Carlos (Segunda Imagen)}
        \subsubsection{Perspectiva}

        \subsubsection{Composición}

        \subsubsection{Clarooscuro}

        \subsubsection{Color}
        \newpage

%-----------------------------------------------------------------
%-----------------------------------------------------------------

    \subsection{16. Jesus}
        \subsubsection{Perspectiva}

        \subsubsection{Composición}

        \subsubsection{Clarooscuro}

        \subsubsection{Color}
        \newpage

%-----------------------------------------------------------------
%-----------------------------------------------------------------

    \subsection{17. Joan (Segunda Imagen)}
        \subsubsection{Perspectiva}

        \subsubsection{Composición}

        \subsubsection{Clarooscuro}

        \subsubsection{Color}
        \newpage

%-----------------------------------------------------------------
%-----------------------------------------------------------------

    \subsection{18. Jesus (Segunda Imagen)}
        \subsubsection{Perspectiva}

        \subsubsection{Composición}

        \subsubsection{Clarooscuro}

        \subsubsection{Color}
        
\end{document}